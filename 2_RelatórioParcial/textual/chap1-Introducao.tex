%Mute warinings 44: User Regex
% chktex-file 44
% chktex-file 1
% chktex-file 2
% chktex-file 8
% chktex-file 13
% chktex-file 12

\chapter{Introdução}\label{chap:Introducao}

A Política Nacional de Recursos Hídricos estabelece que a água é um bem de domínio público, sendo um recurso natural limitado e deve possuir gestão descentralizada, onde participam o ``Poder Público, os usuários e as comunidades'' \cite{LeiPolNacRecHidricos}.

Em 2022 84,9\% da população brasileira era atendida através da rede pública de água, sendo que na região Sudeste esse índice foi de 90,9\% e no Estado de São Paulo 95,2\%. Em um universo de 171 milhões de habitantes, equivale dizer que mais de 25 milhões de pessoas não possuem acesso a distribuição de água por meio da rede pública, necessitando utilizar outros meios para captação, tratamento e eventual distribuição de água potável \cite{MCID_DiagTem2023}.

A legislação brasileira admite e regula o abastecimento de água através das chamadas ``soluções alternativas coletivas de abastecimento de água'', que são modalidades distintas do sistema público, com captação subterrênea ou superficial, com ou sem rede de distribuição \cite{PortCons5}. Ainda que a legislação estabeleça que toda edificação permanente urbana deva estar conectada às redes públicas, admitem-se as soluções alternativas coletivas ou individuais de abastecimento de água \cite{LeiSaneamento}.

Estas soluções alternativas são as possibilidades viáveis a populações que vivem em áreas rurais ou pequenos a médios conglomerados urbanos em regiões periféricas não atendidas pela rede pública de água \cite{ManualSaneamento,AbastAguaPotavel}. No caso dos pequenos conglomerados urbanos estes sistemas são em geral financiados, construídos e administrados pelos próprios habitantes/usuários, com pouca ou nenhuma assistência técnica, como se verá na seção \ref{sec:Delimitacao}.

Em comunidades de médio porte, com de 100 a 200 residências, a administração da distribuição de água, sua medição e cobrança recorre em geral sobre as associações de bairro, dirigidas voluntariamente por moradores eleitos por seus pares, sem o uso de ferramentas que não cadernetas ou quando muito simples planilhas de cálculo.




