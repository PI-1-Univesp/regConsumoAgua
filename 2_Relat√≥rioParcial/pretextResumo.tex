%Mute warinings 44: User Regex
% chktex-file 44
% chktex-file 1
% chktex-file 2
% chktex-file 8
% chktex-file 13
% chktex-file 12
% resumo na língua vernácula (obrigatório)
\setlength{\absparsep}{18pt} % ajusta o espaçamento dos parágrafos do resumo
\SingleSpacing
\begin{fichacatalografica}        
    \noindent
    \SingleSpacing    
        ALVES, Douglas Silva; CAVALCANTE, Carlos Magno da Silva; SANT'ANA, Robert Carvalho; SANTOS, João Matias; SANTOS, Matheus Augusto Matias. \textbf{\imprimirtitulo}. 00f. \imprimirtipotrabalho. Bacharelado em Tecnologia da Informação, Bacharelado em Ciência de Dados e Engenharia da Computação - \textbf{\imprimirinstituicao}. \imprimirorientadorRotulo:  \imprimirorientador. Polos: \imprimirlocal, 2025.
    
    
\end{fichacatalografica}
\begin{resumo}
    Este relatório descreve o desenvolvimento de um software com interface web para administração e publicação de dados em sistemas alternativos coletivos de abastecimento de água.  O sistema foi projetado para auxiliar na gestão do volume de água captado e distribuído, fornecendo um registro histórico dos volumes para a administração da associação de bairro e seus moradores.  O desenvolvimento teve como base a comunidade Recanto do Céu Azul em Mairiporã, SP, e utilizou o ciclo ouvir e interpretar; criar e prototipar; e implementar e testar como paradigma de implementação.  A ferramenta visa minimizar os problemas decorrentes da escassez de água e otimizar a administração desses sistemas.
    
    \noindent
    \textbf{Palavras-chaves}: Abastecimento de água. Framework. Distribuição de água. Sistemas Alternativos de Abastecimento de Água. Rede Pública de Abastecimento de Água.

\end{resumo}